\documentclass[a4paper,12pt]{article}
\usepackage[utf8]{inputenc}
\usepackage{graphicx}
\usepackage{fancyhdr}
\usepackage{xcolor}
\usepackage{listings}
\usepackage[spanish,mexico]{babel}
\usepackage{authblk}

%encabezado
\pagestyle{fancy}
\setlength{\headheight}{2.5cm}  % Ajusta la altura según sea necesario
\fancyhf{}
\fancyhead[L]{Comal}
\fancyhead[R]{Autor: David Mata Guerra \\
	Fecha: \today}
\fancyfoot[C]{\thepage}

\begin{document}
	\section{Comal}
	\subsection{Objetivos}
	El objetivo de éste proyecto es innovar en el gestionamiento actual de la cafetería/restaurante "Comal ++" de la Facultad de Informática. Se busca desarrollar un software el cuál pueda ser utilizado como una herramienta múltiple en tareas que conlleva el Comal++, de manera que llegue a agilizar las tareas a llevar acabo.
	\subsection{Requerimientos}
	\subsubsection{Funcionales}
	\begin{itemize}
		\item Gestión de Menú
		\begin{itemize}
			\item El software debe permitir al usuario administrador crear, visualizar, editar y eliminar platillos del menú.
			\item Los usuarios deben poder visualizar el menú mediante una interfaz visual con información clara y detallada, así como la disponibilidad de un platillo.
		\end{itemize}
		\item Gestión de inventario
		\begin{itemize}
			\item El sistema debe de tener registrado el stock de la materia prima.
			\item El sistema debe de tener registrado información sobre el personal.
			\item El sistema debe tener una modalidad CRUD para el usuario administrador.
		\end{itemize}
		
		\item Gestión de pedidos
		\begin{itemize}
			\item El sistema debe tomar pedidos en tiempo y forma y actualizar el inventario en tiempo real.
			\item Los usuarios deben poder visualizar en que orden va su pedido comparado a los demás.
		\end{itemize}
		
		\item Gestión de usuarios
		\begin{itemize}
			\item El sistema debe poder identificar el modo usuario (administrador, mesero, chef, cliente).
		\end{itemize}
		
		\item Gestión de cortes
		\begin{itemize}
			\item El sistema debe entregar reportes monetarios diarios, semanales y mensuales.
		\end{itemize}
		
	\end{itemize}
	
	\subsection{No Funcionales}
	
	\begin{itemize}
		\item Disponibilidad
		\begin{itemize}
			\item El sistema debe estar disponible para cualquier persona perteneciente a la Facultad de Informática o visitantes.
		\end{itemize}
		\item Usabilidad
		\begin{itemize}
			\item El sistema debe tener una interfaz agradable e intuitiva, haciendo uso de las Heurísticas de Nielsen.
		\end{itemize}
		\item Mantenibilidad
		\begin{itemize}
			\item El sistema debe hacerse de manera que sea fácil de actualizar.
		\end{itemize}
	\end{itemize}
	
	
\end{document}